High energy heavy-ion collisions at RHIC and LHC create a strongly-interacting nuclear matter that exhibit many interesting characteristics. One of which is the collimated emission of particle pairs with small azimuthal-angle separation, $\Delta\phi$, that extends over large range of pseudorapidity differences, $\Delta\eta$. This so called "ridge" correlation was first observed in A+A collisions~\cite{Adare:2008ae, Abelev:2009af, Alver:2009id, ALICE:2011ab, Aad:2012bu, Chatrchyan:2013kba}, but later was also observed in proton-nucleus and light-ion-nucleus collisions~\cite{CMS:2012qk, Abelev:2012ola, Aad:2012gla, Adare:2013piz, Aad:2014lta, Khachatryan:2015waa}, and more recently also observed in high multiplicity proton-proton collisions~\cite{Khachatryan:2010gv, Aad:2015gqa, Aaboud:2016yar, atlas:4}. In A+A collisions, the ridge is believed to be the consequence of collective emission of particles in the azimuthal direction, and the collectivity is generated in the final state after local thermalization, described by relativistic viscous hydrodynamic models. For small system such as $pp$ and $p$+A collisions, the origin of the ridge is less clear. There are considerable debates in the theoretical community on whether the ridge in small systems is of hydrodynamic origin similar to A+A collisions~\cite{Bozek:2013uha} or it is created in the initial state due to gluon saturation effects~\cite{Dusling:2013qoz}.

The ridge signal from two-particle correlation (2PC) is characterized by a Fourier decomposition $\sim 1+2 v_{n}^{2}\text{cos}(n\Delta\phi)$, where the $v_{n}$ denotes the single-particle anisotropy harmonics. The second-order coefficient $v_{2}$ is by far the largest, followed by $v_{3}$. In small collision systems, the extraction of ridge signal requires a careful removal of a large contribution from dijets, which is estimated from 2PC in very low multiplicity events and then subtracted from higher multiplicity events~\cite{Abelev:2012ola, Aad:2012gla, Aad:2014lta, Aad:2015gqa, Aaboud:2016yar, Khachatryan:2016txc}. On the other hand, since collectivity is intrinsically a multi-particle phenomenon, it can be probed more directly using multi-particle correlation (or cumulant) technique~\cite{Borghini:2000sa}. One of the perceived hallmark feature of collectivity is the observation of positively defined signal from $2k$-particle correlation $v_{2}\{2k\},k\ge 2$. The $v_{2}\{4\}$ and $v_{2}\{6\}$ have been measured in high-multiplicity $pp$ and $p$+Pb collisions~\cite{Khachatryan:2015waa, atlas:4, Khachatryan:2016txc, Aad:2013fja}. However, this perception is quite misleading in small collision systems, where the non-flow correlation can be as large or bigger than genuine long-range ridge correlation. Recently, an improved cumulant method based on $\eta$-separated sub-events has been proposed to further reduce the non-flow correlations, in particular from jets  and dijets~\cite{jjia}. The performance of the method in suppressing non-flow correlations has been validated using PYTHIA 8 model, which contains no long-range collective effects.

Multi-particle cumulants suppress jet and short-range correlations, but not completely removing it. In fact $v_{2}\{4\}$ is observed to flip sign at smaller $N_{ch}$, number of charged particle, in $pp$ and $p$+Pb collisions. Recently ATLAS observed that the $N_{ch}$ value where sign-flip happens and the magnitude of $v_{2}\{4\}$ depend also on how the event class are chosen for the calculation of cumulants. In this paper, we show that the choice of event class influences the probability of non-flow, and consequently the non-flow distribution to the $v_{2}\{4\}$. An improved cumulant method is proposed to further suppress non-flow, and therefore reduce the sensitivity to non-flow fluctuations. In this method, cumulants are constructed from particles in several subevents separated in $\eta$, which allows measurement of multi-particle correlations to much lower $N_{ch}$.

The structure of the internal note is as follows. In section~\ref{sec:mtd} subevent cumulant will be briefly introduced, with all the formulas listed. In section~\ref{sec:evtSlc}, data sets together with event and track selections will be described. Detailed procedures of the analysis will be presented in section~\ref{sec:ana}. In section~\ref{sec:sys}, we will discuss all the primary systematics and cross-checks. In section~\ref{sec:result}, all the physics plots in different collision systems will be shown. Section~\ref{sec:summary} summarizes the whole analysis.






