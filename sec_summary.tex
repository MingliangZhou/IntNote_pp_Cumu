The measurement of four-particle cumulant elliptic flow coefficients $C_{2}\{4\}$ for charged particles are presented in 0.17 $\text{pb}^{-1}$ $pp$ data at $\sqrt{s}=5.02$ TeV, 0.9 $\text{pb}^{-1}$ $pp$ data at $\sqrt{s}=13$ TeV and 28 $\text{nb}^{-1}$ $p$+Pb at $\sqrt{s_{NN}}=5.02$ TeV. The $C_{2}\{4\}$ are calculated using traditional cumulant, as well as the recently proposed two-subevent method and three-subevent method, they are presented as a function of $\lr{N_{ch}}$, the average number of charged particles with $p_{\text{T}}>0.4$ GeV. It is found that the $C_{2}\{4\}$ from traditional method is sensitive to the choice of event classes used for averaging, such sensitivity is greatly reduced in the two-subevent method and nearly removed in the three-subevent method, suggesting that the three-subevent method is much less affected by the non-flow effects. Negative $C_{2}\{4\}$ is obtained in all three collision systems in a broad range of $\lr{N_{ch}}$ using the three-subevent method. The magnitude of the $C_{2}\{4\}$ is nearly independent of $\lr{N_{ch}}$ but a slight decrease is observed in $p$+Pb collisions in the large multiplicity region. The single-particle harmonic coefficient $v_{2}\{4\}=(-C_{2}\{4\})^{1/4}$ is calculated and compared with $v_{2}$ obtained previously using a two-particle correlation method, where the non-flow effects was estimated and subtracted. The $v_{2}\{4\}$ is smaller than $v_{2}\{2\}$ as expected for long-range collective ridge, and the difference between $v_{2}\{4\}$ and $v_{2}\{2\}$ are used to infer the number of sources $N_{s}$ in the initial state collision geometry. The $N_{s}$ is found to increase with $\lr{N_{ch}}$ in $p$+Pb collisions and reaches around 20 in highest multiplicity events.


